\documentclass[12pt,a4paper]{article}
\usepackage[margin=1in]{geometry} % 设置页面边距
\usepackage{ctex}
\usepackage{graphicx}
\usepackage{float}
\usepackage{amsmath}
\usepackage{amssymb} % 数学符号
\usepackage{booktabs} % 绘制漂亮的表格
\usepackage{array} % 数组和表格
\usepackage{multirow} % 多行合并的表格
\usepackage{caption} % 设置图片和表格的标题格式
\usepackage{subcaption} % 多个子图或表格
\usepackage{parskip}
\setlength{\parindent}{0pt}

\title{\textbf{搜索实践实验报告}}
\author{张立博\ 未央-能动11\ 2021012487}
\date{2023.4.1}

\begin{document}

\maketitle

\section*{问题一}
以下根据棋盘的大小简要描述AI的表现
\subsection{$3\times 3$}
无论AI先后手,都可以找到不输的策略,但较大概率发生平局
\subsection{$3\times 4$}
若AI先手则必胜
\subsection{$9\times 9$}
\subsubsection*{截断搜索+评估函数}
依据评估函数AI已具有一定的棋力,经测试与一般人类对弈AI无论先后手均具有一定胜率
\subsubsection*{朴素MCST}
行棋较慢且落子位置分散,不能展现出很强的棋力
\subsubsection*{AlphaZero}
引入评估函数后AI具有一定的棋力,但相较于使用截断搜索的AI行棋速度较慢

\section*{问题二}
朴素minimax几乎无法落子,经测试在15min内无法给出第一步,遂终止\\
alpha-beta行棋速度显著快于朴素minimax,第一步落子约需要3s,后续落子速度更快,能在0.5s内进行落子
\section*{问题三}
对于每一个玩家,分别记录其活四、冲四、活三、冲三、活二的数量以及最远棋子距离棋盘中心的相对距离\\
然后玩家的属性进行权值估计,同时需要除以常数并更新分数,保证得分在$[-1,1]$中\\
根据先验知识,权值活四>冲四>活三>距离>冲三>活二;同时由于敌方的冲四可以进行防守,所以敌方冲四的权值显著低于己方冲四的权值\\
具体地,若玩家为智能体:
\begin{align*}
    score += (40000*live\_four + 5000*four + 1000*live\_three + 100 * three + 50 *live\_two - 500 * distance)/50000
\end{align*}
若玩家为对手:
\begin{align*}
    score -= (40000*live\_four + 100*four + 1000*live\_three + 100 * three + 50 *live\_two - 500 * distance)/50000
\end{align*}
\section*{问题四}
对战结果:alpha-beta完全胜于MCTS\\
原因分析:alpha-beta在探索一层后采用根据先验知识设计的评估函数返回最佳的行动,而MCTS在进行模拟时采用完全随机的方式且模拟测试数量有限,无法得到合理的行棋策略,效果较差
\section*{问题五}
AlphaZero相较于MCTS行棋更合理,通过二者对局发现前者棋力明显大于后者,这是因为在模拟中AlphaZero采用评估函数的方式代替了随机游戏
\end{document}